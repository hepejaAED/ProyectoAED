%  LaTeX support: latex@mdpi.com
%  For support, please attach all files needed for compiling as well as the log file, and specify your operating system, LaTeX version, and LaTeX editor.

%=================================================================
% pandoc conditionals added to preserve backwards compatibility with previous versions of rticles

\documentclass[notspecified,article,submit,moreauthors,pdftex]{Definitions/mdpi}


%% Some pieces required from the pandoc template
\setlist[itemize]{leftmargin=*,labelsep=5.8mm}
\setlist[enumerate]{leftmargin=*,labelsep=4.9mm}


%--------------------
% Class Options:
%--------------------
%----------
% journal
%----------
% Choose between the following MDPI journals:
% acoustics, actuators, addictions, admsci, adolescents, aerobiology, aerospace, agriculture, agriengineering, agrochemicals, agronomy, ai, air, algorithms, allergies, alloys, analytica, analytics, anatomia, animals, antibiotics, antibodies, antioxidants, applbiosci, appliedchem, appliedmath, applmech, applmicrobiol, applnano, applsci, aquacj, architecture, arm, arthropoda, arts, asc, asi, astronomy, atmosphere, atoms, audiolres, automation, axioms, bacteria, batteries, bdcc, behavsci, beverages, biochem, bioengineering, biologics, biology, biomass, biomechanics, biomed, biomedicines, biomedinformatics, biomimetics, biomolecules, biophysica, biosensors, biotech, birds, bloods, blsf, brainsci, breath, buildings, businesses, cancers, carbon, cardiogenetics, catalysts, cells, ceramics, challenges, chemengineering, chemistry, chemosensors, chemproc, children, chips, cimb, civileng, cleantechnol, climate, clinpract, clockssleep, cmd, coasts, coatings, colloids, colorants, commodities, compounds, computation, computers, condensedmatter, conservation, constrmater, cosmetics, covid, crops, cryptography, crystals, csmf, ctn, curroncol, cyber, dairy, data, ddc, dentistry, dermato, dermatopathology, designs, devices, diabetology, diagnostics, dietetics, digital, disabilities, diseases, diversity, dna, drones, dynamics, earth, ebj, ecologies, econometrics, economies, education, ejihpe, electricity, electrochem, electronicmat, electronics, encyclopedia, endocrines, energies, eng, engproc, entomology, entropy, environments, environsciproc, epidemiologia, epigenomes, est, fermentation, fibers, fintech, fire, fishes, fluids, foods, forecasting, forensicsci, forests, foundations, fractalfract, fuels, future, futureinternet, futurepharmacol, futurephys, futuretransp, galaxies, games, gases, gastroent, gastrointestdisord, gels, genealogy, genes, geographies, geohazards, geomatics, geosciences, geotechnics, geriatrics, grasses, gucdd, hazardousmatters, healthcare, hearts, hemato, hematolrep, heritage, higheredu, highthroughput, histories, horticulturae, hospitals, humanities, humans, hydrobiology, hydrogen, hydrology, hygiene, idr, ijerph, ijfs, ijgi, ijms, ijns, ijpb, ijtm, ijtpp, ime, immuno, informatics, information, infrastructures, inorganics, insects, instruments, inventions, iot, j, jal, jcdd, jcm, jcp, jcs, jcto, jdb, jeta, jfb, jfmk, jimaging, jintelligence, jlpea, jmmp, jmp, jmse, jne, jnt, jof, joitmc, jor, journalmedia, jox, jpm, jrfm, jsan, jtaer, jvd, jzbg, kidneydial, kinasesphosphatases, knowledge, land, languages, laws, life, liquids, literature, livers, logics, logistics, lubricants, lymphatics, machines, macromol, magnetism, magnetochemistry, make, marinedrugs, materials, materproc, mathematics, mca, measurements, medicina, medicines, medsci, membranes, merits, metabolites, metals, meteorology, methane, metrology, micro, microarrays, microbiolres, micromachines, microorganisms, microplastics, minerals, mining, modelling, molbank, molecules, mps, msf, mti, muscles, nanoenergyadv, nanomanufacturing,\gdef\@continuouspages{yes}} nanomaterials, ncrna, ndt, network, neuroglia, neurolint, neurosci, nitrogen, notspecified, %%nri, nursrep, nutraceuticals, nutrients, obesities, oceans, ohbm, onco, %oncopathology, optics, oral, organics, organoids, osteology, oxygen, parasites, parasitologia, particles, pathogens, pathophysiology, pediatrrep, pharmaceuticals, pharmaceutics, pharmacoepidemiology,\gdef\@ISSN{2813-0618}\gdef\@continuous pharmacy, philosophies, photochem, photonics, phycology, physchem, physics, physiologia, plants, plasma, platforms, pollutants, polymers, polysaccharides, poultry, powders, preprints, proceedings, processes, prosthesis, proteomes, psf, psych, psychiatryint, psychoactives, publications, quantumrep, quaternary, qubs, radiation, reactions, receptors, recycling, regeneration, religions, remotesensing, reports, reprodmed, resources, rheumato, risks, robotics, ruminants, safety, sci, scipharm, sclerosis, seeds, sensors, separations, sexes, signals, sinusitis, skins, smartcities, sna, societies, socsci, software, soilsystems, solar, solids, spectroscj, sports, standards, stats, std, stresses, surfaces, surgeries, suschem, sustainability, symmetry, synbio, systems, targets, taxonomy, technologies, telecom, test, textiles, thalassrep, thermo, tomography, tourismhosp, toxics, toxins, transplantology, transportation, traumacare, traumas, tropicalmed, universe, urbansci, uro, vaccines, vehicles, venereology, vetsci, vibration, virtualworlds, viruses, vision, waste, water, wem, wevj, wind, women, world, youth, zoonoticdis 
% For posting an early version of this manuscript as a preprint, you may use "preprints" as the journal. Changing "submit" to "accept" before posting will remove line numbers.

%---------
% article
%---------
% The default type of manuscript is "article", but can be replaced by: 
% abstract, addendum, article, book, bookreview, briefreport, casereport, comment, commentary, communication, conferenceproceedings, correction, conferencereport, entry, expressionofconcern, extendedabstract, datadescriptor, editorial, essay, erratum, hypothesis, interestingimage, obituary, opinion, projectreport, reply, retraction, review, perspective, protocol, shortnote, studyprotocol, systematicreview, supfile, technicalnote, viewpoint, guidelines, registeredreport, tutorial
% supfile = supplementary materials

%----------
% submit
%----------
% The class option "submit" will be changed to "accept" by the Editorial Office when the paper is accepted. This will only make changes to the frontpage (e.g., the logo of the journal will get visible), the headings, and the copyright information. Also, line numbering will be removed. Journal info and pagination for accepted papers will also be assigned by the Editorial Office.

%------------------
% moreauthors
%------------------
% If there is only one author the class option oneauthor should be used. Otherwise use the class option moreauthors.

%---------
% pdftex
%---------
% The option pdftex is for use with pdfLaTeX. Remove "pdftex" for (1) compiling with LaTeX & dvi2pdf (if eps figures are used) or for (2) compiling with XeLaTeX.

%=================================================================
% MDPI internal commands - do not modify
\firstpage{1} 
\makeatletter 
\setcounter{page}{\@firstpage} 
\makeatother
\pubvolume{1}
\issuenum{1}
\articlenumber{0}
\pubyear{2024}
\copyrightyear{2024}
%\externaleditor{Academic Editor: Firstname Lastname}
\datereceived{ } 
\daterevised{ } % Comment out if no revised date
\dateaccepted{ } 
\datepublished{ } 
%\datecorrected{} % For corrected papers: "Corrected: XXX" date in the original paper.
%\dateretracted{} % For corrected papers: "Retracted: XXX" date in the original paper.
\hreflink{https://doi.org/} % If needed use \linebreak
%\doinum{}
%\pdfoutput=1 % Uncommented for upload to arXiv.org
%\CorrStatement{yes}  % For updates


%=================================================================
% Add packages and commands here. The following packages are loaded in our class file: fontenc, inputenc, calc, indentfirst, fancyhdr, graphicx, epstopdf, lastpage, ifthen, float, amsmath, amssymb, lineno, setspace, enumitem, mathpazo, booktabs, titlesec, etoolbox, tabto, xcolor, colortbl, soul, multirow, microtype, tikz, totcount, changepage, attrib, upgreek, array, tabularx, pbox, ragged2e, tocloft, marginnote, marginfix, enotez, amsthm, natbib, hyperref, cleveref, scrextend, url, geometry, newfloat, caption, draftwatermark, seqsplit
% cleveref: load \crefname definitions after \begin{document}

%=================================================================
% Please use the following mathematics environments: Theorem, Lemma, Corollary, Proposition, Characterization, Property, Problem, Example, ExamplesandDefinitions, Hypothesis, Remark, Definition, Notation, Assumption
%% For proofs, please use the proof environment (the amsthm package is loaded by the MDPI class).

%=================================================================
% Full title of the paper (Capitalized)
\Title{Evolución de la actividad, ocupación para grupos de edad mayores
de 16 años}

% MDPI internal command: Title for citation in the left column
\TitleCitation{Evolución de la actividad, ocupación para grupos de edad
mayores de 16 años}

% Author Orchid ID: enter ID or remove command
%\newcommand{\orcidauthorA}{0000-0000-0000-000X} % Add \orcidA{} behind the author's name
%\newcommand{\orcidauthorB}{0000-0000-0000-000X} % Add \orcidB{} behind the author's name


% Authors, for the paper (add full first names)
\Author{Jeremy Joel Aguilar Marin$^{}$, Jose Aguilar Milla$^{}$, Javier
Herrero Pérez$^{}$}


%\longauthorlist{yes}


% MDPI internal command: Authors, for metadata in PDF
\AuthorNames{Jeremy Joel Aguilar Marin, Jose Aguilar Milla, Javier
Herrero Pérez}

% MDPI internal command: Authors, for citation in the left column

% Affiliations / Addresses (Add [1] after \address if there is only one affiliation.)
\address{%
}

% Contact information of the corresponding author
\corres{Correspondence: }

% Current address and/or shared authorship








% The commands \thirdnote{} till \eighthnote{} are available for further notes

% Simple summary
\simplesumm{A Simple summary goes here.}

%\conference{} % An extended version of a conference paper

% Abstract (Do not insert blank lines, i.e. \\)


% Keywords
\keyword{keyword 1; keyword 2; keyword 3 (list three to ten pertinent
keywords specific to the article, yet reasonably common within the
subject discipline.).}

% The fields PACS, MSC, and JEL may be left empty or commented out if not applicable
%\PACS{J0101}
%\MSC{}
%\JEL{}

%%%%%%%%%%%%%%%%%%%%%%%%%%%%%%%%%%%%%%%%%%
% Only for the journal Diversity
%\LSID{\url{http://}}

%%%%%%%%%%%%%%%%%%%%%%%%%%%%%%%%%%%%%%%%%%
% Only for the journal Applied Sciences

%%%%%%%%%%%%%%%%%%%%%%%%%%%%%%%%%%%%%%%%%%

%%%%%%%%%%%%%%%%%%%%%%%%%%%%%%%%%%%%%%%%%%
% Only for the journal Data



%%%%%%%%%%%%%%%%%%%%%%%%%%%%%%%%%%%%%%%%%%
% Only for the journal Toxins


%%%%%%%%%%%%%%%%%%%%%%%%%%%%%%%%%%%%%%%%%%
% Only for the journal Encyclopedia


%%%%%%%%%%%%%%%%%%%%%%%%%%%%%%%%%%%%%%%%%%
% Only for the journal Advances in Respiratory Medicine
%\addhighlights{yes}
%\renewcommand{\addhighlights}{%

%\noindent This is an obligatory section in “Advances in Respiratory Medicine”, whose goal is to increase the discoverability and readability of the article via search engines and other scholars. Highlights should not be a copy of the abstract, but a simple text allowing the reader to quickly and simplified find out what the article is about and what can be cited from it. Each of these parts should be devoted up to 2~bullet points.\vspace{3pt}\\
%\textbf{What are the main findings?}
% \begin{itemize}[labelsep=2.5mm,topsep=-3pt]
% \item First bullet.
% \item Second bullet.
% \end{itemize}\vspace{3pt}
%\textbf{What is the implication of the main finding?}
% \begin{itemize}[labelsep=2.5mm,topsep=-3pt]
% \item First bullet.
% \item Second bullet.
% \end{itemize}
%}


%%%%%%%%%%%%%%%%%%%%%%%%%%%%%%%%%%%%%%%%%%


% tightlist command for lists without linebreak
\providecommand{\tightlist}{%
  \setlength{\itemsep}{0pt}\setlength{\parskip}{0pt}}


% Add imagehandling
% Custom command from Pandoc 3.2.1 to scale images if necessary
\usepackage{graphicx}
\makeatletter
\newsavebox\pandoc@box
\newcommand*\pandocbounded[1]{% scales image to fit in text height/width
  \sbox\pandoc@box{#1}%
  \Gscale@div\@tempa{\textheight}{\dimexpr\ht\pandoc@box+\dp\pandoc@box\relax}%
  \Gscale@div\@tempb{\linewidth}{\wd\pandoc@box}%
  \ifdim\@tempb\p@<\@tempa\p@\let\@tempa\@tempb\fi% select the smaller of both
  \ifdim\@tempa\p@<\p@\scalebox{\@tempa}{\usebox\pandoc@box}%
  \else\usebox{\pandoc@box}%
  \fi%
}
\makeatother



\usepackage{longtable}

\begin{document}



%%%%%%%%%%%%%%%%%%%%%%%%%%%%%%%%%%%%%%%%%%

\section{Introducción}\label{introducciuxf3n}

Este artículo presenta un enfoque en R para el Análisis Exploratorio de
Datos de la Encuesta de Población Activa (EPA), centrándose en la
evolución histórica de la brecha de género en el mercado laboral
español. Utilizando el ecosistema \texttt{tidyverse}, el estudio
comienza con una limpieza y transformación de los datos empleando
\texttt{dplyr} y el cálculo de Tasa de paro, actividad y ocupación,
claves para el análisis. Seguido de la visualización con
\texttt{ggplot2} para ilustrar la evolución temporal. Por último, se
validan los resultados con pruebas de \emph{t Student} para confirmar la
significancia estadística de los resutlados.

\subsection{Carga de Librerías}\label{carga-de-libreruxedas}

\section{Exploración inicial}\label{exploraciuxf3n-inicial}

\subsection{Importación y verificación
inicial}\label{importaciuxf3n-y-verificaciuxf3n-inicial}

\begin{itemize}
\tightlist
\item
  \textbf{Codificación e importación}: Se verifica la codificación y se
  importa el archivo CSV delimitado por tabulaciones, se indica que el
  separador decimal es la coma y que los valores \emph{NA} están
  codificados como \texttt{".."}
\end{itemize}

\begin{verbatim}
## # A tibble: 3 x 2
##   encoding   confidence
##   <chr>           <dbl>
## 1 UTF-8            1   
## 2 ISO-8859-1       0.98
## 3 ISO-8859-2       0.28
\end{verbatim}

\begin{verbatim}
## # A tibble: 6 x 5
##   Sexo        Edad  `Relación con la actividad económica` Periodo  Total
##   <chr>       <chr> <chr>                                   <dbl>  <dbl>
## 1 Ambos sexos Total Total                                    2024 41566.
## 2 Ambos sexos Total Total                                    2023 40983.
## 3 Ambos sexos Total Total                                    2022 40367.
## 4 Ambos sexos Total Total                                    2021 39926.
## 5 Ambos sexos Total Total                                    2020 39579.
## 6 Ambos sexos Total Total                                    2019 39269.
\end{verbatim}

\subsection{Valores faltantes}\label{valores-faltantes}

Todos los valores ausentes se encuentran en \texttt{Total}

\pandocbounded{\includegraphics[keepaspectratio]{ProyectoAED_files/figure-latex/unnamed-chunk-6-1.pdf}}

Se observa que los valores faltantes vienen asociados a grupos de edad
altos y cuya relación con la actividad económica es ``Parados que buscan
primer empleo'' o ``Parados''.

\section{Definición de variables}\label{definiciuxf3n-de-variables}

El valor de la columna \textbf{Total} viene definido con la siguiente
expresión:

\[\text{Total}=\text{Activos}+\text{Inactivos}\] Y los \textbf{Activos}
se dividen en Ocupados y Parados:

\[\text{Activos}=\text{Ocupados}+\text{Parados}\] \#\# Fórmulas para el
análisis (Tasas)

Las tasas permiten normalizar los datos y son la base para el análisis
de brechas y series temporales:

La \textbf{tasa de paro} mide la proporción de la población activa que
está desempleada:

\[\text{Tasa de paro}=\frac{\text{Parados}}{\text{Activos}} \cdot 100\]

La \textbf{tasa de actividad} mide la participación de una población en
el mercado laboral:

\[\text{Tasa de actividad}=\frac{\text{Activos}}{\text{Total}} \cdot 100\]
La \textbf{tasa de ocupación} muestra qué parte de la población total de
referencia tiene un empleo:

\[\text{Tasa de ocupación}=\frac{\text{Ocupados}}{\text{Total}} \cdot 100\]

\section{Limpieza}\label{limpieza}

\begin{itemize}
\item
  \textbf{Renombrar columnas}: Utilizando \texttt{rename()} para acortar
  \texttt{"Relación\ con\ la\ actividad\ económica"} por
  \texttt{"Situacion\_Laboral"} para simplificar el manejo de columnas.
\item
  \textbf{Filtrado de doble conteo}: Se eliminan las instancias donde
  las variables representan la suma de otras instancias
  (\texttt{Sexo="Ambos\ sexos"} y \texttt{Edad="Total"}) para evitar
  doble conteo y así verificar que cada registro sea una observación
  única.
\item
  \textbf{Se aplica \texttt{pivot\_wider()}} utilizando los valores de
  la columna \texttt{Situacion\_Laboral} para crear las nuevas columnas
  (\texttt{names\_from}) y los valores numéricos de la columna
  \texttt{Total}.
\item
  \textbf{Renombrar variables}: Se renombra la columna
  \texttt{"parados\ que\ buscan\ primer\ empleo"} a un formato más corto
  Parados\_sin\_exp
\item
  \textbf{Imputación de NA a 0}: Los valores faltantes se concentran en
  categorías como Personas mayores de 65 en paro o que buscan su primer
  empleo. El INE suprime el dato por ser un valor insignificante, por lo
  que se opta por remplazar los valores nulos a 0 para no perder el
  resto de información válida de la instancia.
\item
  \textbf{Creación de tasas}: Se añaden las columnas
  \texttt{Tasa\_paro}, \texttt{Tasa\_actividad} y
  \texttt{Tasa\_ocupacion} utilizando \texttt{mutate()} y las
  expresiones de la EPA.
\item
  \textbf{Ordenación en factores}: Se ordena la columna \texttt{Edad}
  como un factor ordinal y la columna \texttt{Sexo} como factor nominal.
\end{itemize}

Se comprueba que la limpieza esté bien hecha, primero viendo si el
formato de los datos es el adecuado y asegurando que no quedan valores
faltantes.

\section{Análisis de las tasas}\label{anuxe1lisis-de-las-tasas}

\subsection{Distribución de la tasa de actividad por
sexo}\label{distribuciuxf3n-de-la-tasa-de-actividad-por-sexo}

\pandocbounded{\includegraphics[keepaspectratio]{ProyectoAED_files/figure-latex/unnamed-chunk-9-1.pdf}}

\subsection{Evolucion de la tasa de paro por género y evolución de la
brecha de la tasa de paro por
género}\label{evolucion-de-la-tasa-de-paro-por-guxe9nero-y-evoluciuxf3n-de-la-brecha-de-la-tasa-de-paro-por-guxe9nero}

En este bloque de código se construyen varias representaciones gráficas
con el fin de analizar la evolución de la tasa de paro y la brecha de
género a lo largo del tiempo. Para ello, se parte de una tabla en la que
se calcula la variable \emph{brecha\_paro}, definida como la diferencia
entre la tasa de paro de mujeres y la de hombres. Posteriormente, los
datos se agrupan por periodo para obtener las tasas medias anuales de
paro por sexo y la brecha media correspondiente.

Agrupación de datos por periodo y por sexo

\pandocbounded{\includegraphics[keepaspectratio]{ProyectoAED_files/figure-latex/unnamed-chunk-11-1.pdf}}

La gráfica enseña la evolución de la tasa de actividad en España entre
los años 2006 y 2024, diferenciada por sexo. Por un lado, la proporción
de hombres es mucho mayor a la de mujeres, cosa que va ligada a que los
hombres siempre han tenido mayor participicaion en muchos mas sectores,
como lo puede ser la construcción, y a que las mujeres se ligan mas al
cuidado familiar. Pese a esto, la actividad masculina en 2008 se
desacelera economicamnete, esto provocado por la crisis financiera que
se vivió en la epoca. Es en 2017/18 que comienza su recuperación, pero
nunca a los niveles previos a la crisis. En 2020, debido al Covid-19, se
observa otra caida en la tendencía (porque el Covid-19 afectó a muchos
sectores, como la industría por ejemplo). Aunque se observa también su
recuperació, tampoco volvió a los nieveles previos de la crisis de 2008
por el otro lado, la tasa de actividad femenina se muestra, en el
periodo, ascendente ya que han ido incorporandose cada vez mas en el
mercado laboral español

Tasa de actividad por edad y sexo

\pandocbounded{\includegraphics[keepaspectratio]{ProyectoAED_files/figure-latex/unnamed-chunk-13-1.pdf}}

\pandocbounded{\includegraphics[keepaspectratio]{ProyectoAED_files/figure-latex/unnamed-chunk-14-1.pdf}}

\pandocbounded{\includegraphics[keepaspectratio]{ProyectoAED_files/figure-latex/unnamed-chunk-15-1.pdf}}
El primer gráfico (\texttt{evo\_brecha\_total}) representa la evolución
de la brecha de paro total tanto con lineas como con puntos para mostrar
la tendencia y los valores anuales. En él se observa cómo la diferencia
entre hombres y mujeres en la tasa de paro varía a lo largo del tiempo,
con un aumento notable en los años posteriores a la crisis de 2008 y una
ligera reducción en la última etapa analizada. Por su parte, el segundo
gráfico (\texttt{evo\_tasa\_total}) muestra la evolución de la tasa de
paro de hombres y mujeres de forma independiente, utilizando colores
diferenciados para cada grupo. Esta visualización permite comparar
directamente la evolución de ambos sexos, destacando la persistencia de
una mayor tasa de paro entre las mujeres en prácticamente todos los años
del periodo estudiado.

Por último, el tercer gráfico (\texttt{evo\_tasa\_y\_brecha}) combina
ambas perspectivas en una única figura. Se incluyen las líneas de
hombres, mujeres y brecha, diferenciadas mediante colores específicos
definidos con \texttt{scale\_color\_manual()}. Este enfoque conjunto
facilita observar la relación entre las dos tasas y su diferencia a lo
largo del tiempo, ofreciendo una visión global de la evolución de las
desigualdades de género en el desempleo. En conjunto, el código y las
representaciones resultantes permiten analizar de forma visual y
comparativa la magnitud y la persistencia de la brecha de género en el
mercado laboral español.

\subsection{Evolucion de la tasa de paro y su brecha por grupo de
edad}\label{evolucion-de-la-tasa-de-paro-y-su-brecha-por-grupo-de-edad}

En este bloque de código se elaboran una serie de gráficos exploratorios
con el objetivo de analizar la evolución de la brecha de paro entre
mujeres y hombres desagregada por grupos de edad. A partir del conjunto
de datos principal, se construye una tabla denominada brecha\_edades que
contiene la tasa de paro por sexo y edad, y se calcula la variable
brecha\_paro como la diferencia entre las tasas de mujeres y hombres.
Posteriormente, los grupos de edad se clasifican en intervalos de diez
años, y los datos se dividen en distintos subconjuntos (df\_1, df\_2,
df\_3) para facilitar la representación y mejorar la legibilidad de los
gráficos.

\subsection{Gráfica de Barras comparación grupos de edades con mayor
brecha en 2006 y en
2024}\label{gruxe1fica-de-barras-comparaciuxf3n-grupos-de-edades-con-mayor-brecha-en-2006-y-en-2024}

En este bloque de código se elaboran una serie de gráficos exploratorios
de barras destinados a analizar la evolución de la brecha de paro por
grupos de edad a lo largo del tiempo. Para ello, a partir del conjunto
brecha\_edades, que contiene la diferencia entre las tasas de paro
femeninas y masculinas, se genera un conjunto de gráficos individuales
por año mediante la función lapply(). En cada uno de ellos, las barras
representan la magnitud de la brecha en los distintos grupos de edad,
mientras que la línea discontinua negra indica la media general de la
brecha para ese año. De esta forma, el código permite observar en
paralelo la estructura intergeneracional de las desigualdades laborales
y cómo varía la media de brecha entre periodos.

\pandocbounded{\includegraphics[keepaspectratio]{ProyectoAED_files/figure-latex/unnamed-chunk-17-1.pdf}}
Los gráficos resultantes muestran, de forma clara y comparativa, cómo la
intensidad de la brecha de género en el desempleo varía según la edad y
el momento temporal. En los primeros años analizados, la brecha es
especialmente elevada entre los jóvenes, lo que evidencia una mayor
vulnerabilidad del empleo femenino en los primeros tramos de edad. En
cambio, en los años más recientes para esos tramos de edad, las
diferencias tienden a moderarse y presentan un patrón más equilibrado
entre grupos, aunque persisten desigualdades estructurales en los
segmentos intermedios y de mayor edad.

En conjunto, este bloque busca ofrecer una visión desagregada y
evolutiva de la brecha de paro por edad. Gracias a la representación
mediante diagramas de barras, se facilita la comparación entre grupos
generacionales y se identifican con claridad los grupos que más
contribuyen a mantener las disparidades laborales entre hombres y
mujeres a lo largo del tiempo.

\pandocbounded{\includegraphics[keepaspectratio]{ProyectoAED_files/figure-latex/unnamed-chunk-19-1.pdf}}

\section{Distribución de la población en el mercado
laboral}\label{distribuciuxf3n-de-la-poblaciuxf3n-en-el-mercado-laboral}

\subsection{Composición del mercado
laboral}\label{composiciuxf3n-del-mercado-laboral}

En este bloque de código se genera un gráfico exploratorio con el
propósito de comprobar la visualización de la composición del mercado
laboral antes de automatizar el proceso para todos los años. Para ello,
se filtran los datos del año 2024 y se seleccionan las variables
relevantes relacionadas con el estado laboral (\emph{Activos, Ocupados,
Parados e Inactivos}), junto con la edad y el sexo. Posteriormente,
mediante la función \texttt{pivot\_longer()}, se reorganiza la tabla a
formato largo, lo que permite representar las distintas categorías
laborales como valores dentro de una misma variable (\emph{Estado}),
facilitando así la creación de un gráfico apilado en \emph{ggplot2}.

El gráfico resultante muestra la distribución proporcional de la
población según su situación laboral y grupo de edad, diferenciando
entre hombres y mujeres mediante
\texttt{facet\_wrap(\textasciitilde{}Sexo)}. Se emplea
\texttt{geom\_col(position\ =\ "fill")} para que las barras representen
proporciones y no valores absolutos, y se aplican ajustes estéticos para
mejorar la legibilidad, como la rotación de etiquetas del eje X y el uso
de \texttt{theme\_minimal()}. Este gráfico sirvió como prueba inicial
para evaluar la claridad y estructura de la visualización antes de
desarrollar la función que posteriormente generó automáticamente el
mismo tipo de gráfico para todos los años del periodo analizado.

Tras la comprobación inicial con los datos de 2024, se elaboró este
bloque de código para automatizar la generación del mismo tipo de
gráfico para todos los años del conjunto de datos. Para ello, se utiliza
la función \texttt{lapply()}, que aplica el mismo procedimiento de
visualización a cada año contenido en la variable \texttt{años}. Dentro
de la función, los datos se filtran por periodo y se representa la
composición del mercado laboral mediante
\texttt{geom\_col(position\ =\ "fill")}, mostrando la proporción de cada
estado laboral (ocupados, parados e inactivos) dentro de cada grupo de
edad.

El código utiliza \texttt{facet\_wrap(\textasciitilde{}Sexo)} para
mostrar en paneles separados la información correspondiente a hombres y
mujeres. El título de cada gráfico se personaliza dinámicamente con el
año correspondiente mediante \texttt{paste()}. Este procedimiento
permitió obtener una serie de gráficos anuales, facilitando así el
análisis comparativo de la evolución de la estructura del mercado
laboral a lo largo del tiempo y entre sexos.

\begin{verbatim}
## [[1]]
\end{verbatim}

\pandocbounded{\includegraphics[keepaspectratio]{ProyectoAED_files/figure-latex/unnamed-chunk-20-1.pdf}}

\begin{verbatim}
## 
## [[2]]
\end{verbatim}

\pandocbounded{\includegraphics[keepaspectratio]{ProyectoAED_files/figure-latex/unnamed-chunk-20-2.pdf}}

Del análisis de los gráficos se observa una estructura bastante estable
del mercado laboral por edad y sexo a lo largo del tiempo, con ligeras
variaciones asociadas a los ciclos económicos. En ambos sexos, el grupo
central de edad (30-54 años) concentra la mayor proporción de personas
ocupadas, mientras que los extremos ---jóvenes y mayores--- presentan
una mayor presencia de inactivos. Las mujeres muestran sistemáticamente
una menor proporción de ocupación y una mayor de inactividad en
comparación con los hombres, aunque esta diferencia tiende a reducirse
en los años más recientes. Asimismo, la proporción de personas paradas
es más significativa en los grupos jóvenes, reflejando una mayor
vulnerabilidad del empleo juvenil. En conjunto, los gráficos ponen de
manifiesto que, pese a los avances en participación femenina y la
estabilidad general del empleo en edades medias, persisten desigualdades
de género e importantes diferencias intergeneracionales dentro del
mercado laboral.

\#\#Proporción de parados En este bloque de código se genera una serie
de gráficos que representan la distribución del paro dentro de cada sexo
por grupo de edad y año. Para ello, se utiliza una estructura, que
permite crear automáticamente un gráfico por cada año disponible en el
conjunto de datos. Primero, los datos se filtran por periodo y se
agrupan por sexo y año, tras lo cual se calcula la proporción que
representa cada grupo de edad dentro del total de personas paradas del
mismo sexo (\texttt{Parados\ /\ sum(Parados)}).

A continuación, se construyen los gráficos con ggplot2, empleando
\texttt{geom\_col(position\ =\ "dodge")} para comparar visualmente la
distribución entre hombres y mujeres dentro de cada grupo de edad. El
objetivo de este análisis es observar cómo se reparte el desempleo
dentro de cada sexo a lo largo del tiempo, identificando posibles
diferencias estructurales entre hombres y mujeres en función de la edad.
De esta manera, se obtiene una visión más detallada del peso relativo de
cada grupo de edad en el total de personas paradas de cada sexo y su
evolución temporal.

En conjunto, estas gráficas ponen de manifiesto un proceso de
envejecimiento del desempleo y una convergencia progresiva entre hombres
y mujeres en la estructura interna del paro, aunque todavía persisten
diferencias derivadas de los roles laborales y sociales tradicionales.
Mientras los hombres mayores ganan peso dentro del paro total por el
retraso en la jubilación, las mujeres siguen mostrando una mayor
vulnerabilidad en las edades centrales, reflejando las desigualdades de
género que aún caracterizan el mercado laboral español.

\#Distribución de ocupados e inactivos por sexo

\pandocbounded{\includegraphics[keepaspectratio]{ProyectoAED_files/figure-latex/unnamed-chunk-22-1.pdf}}

\section{act inact}\label{act-inact}

AQUI HAY QUE CALCULAR EL COCIENTE ENTRE ACTIVOS E INACTIVOS

\section{Resultados}\label{resultados}

\subsection{desv est y varianza}\label{desv-est-y-varianza}

\begin{verbatim}
## # A tibble: 2 x 3
##   Sexo    sd_ocupados var_ocupados
##   <fct>         <dbl>        <dbl>
## 1 Hombres        591.      349261.
## 2 Mujeres        490.      240316.
\end{verbatim}

\begin{verbatim}
## # A tibble: 1 x 2
##   brecha_ocup brecha_inact
##         <dbl>        <dbl>
## 1       -166.         247.
\end{verbatim}

\begin{verbatim}
## 
##  Welch Two Sample t-test
## 
## data:  Tasa_ocupacion by Sexo
## t = 3.767, df = 7.4, p-value = 0.00633
## alternative hypothesis: true difference in means between group Hombres and group Mujeres is not equal to 0
## 95 percent confidence interval:
##   3.368217 14.401655
## sample estimates:
## mean in group Hombres mean in group Mujeres 
##              82.77577              73.89084
\end{verbatim}

%%%%%%%%%%%%%%%%%%%%%%%%%%%%%%%%%%%%%%%%%%

\vspace{6pt}

%%%%%%%%%%%%%%%%%%%%%%%%%%%%%%%%%%%%%%%%%%
%% optional

% Only for the journal Methods and Protocols:
% If you wish to submit a video article, please do so with any other supplementary material.
% \supplementary{The following supporting information can be downloaded at: \linksupplementary{s1}, Figure S1: title; Table S1: title; Video S1: title. A supporting video article is available at doi: link.}

% Only for journal Hardware:
% If you wish to submit a video article, please do so with any other supplementary material.
% \supplementary{The following supporting information can be downloaded at: \linksupplementary{s1}, Figure S1: title; Table S1: title; Video S1: title.\vspace{6pt}\\
%\begin{tabularx}{\textwidth}{lll}
%\toprule
%\textbf{Name} & \textbf{Type} & \textbf{Description} \\
%\midrule
%S1 & Python script (.py) & Script of python source code used in XX \\
%S2 & Text (.txt) & Script of modelling code used to make Figure X \\
%S3 & Text (.txt) & Raw data from experiment X \\
%S4 & Video (.mp4) & Video demonstrating the hardware in use \\
%... & ... & ... \\
%\bottomrule
%\end{tabularx}
%}

%%%%%%%%%%%%%%%%%%%%%%%%%%%%%%%%%%%%%%%%%%





% Only for journal Nursing Reports
%\publicinvolvement{Please describe how the public (patients, consumers, carers) were involved in the research. Consider reporting against the GRIPP2 (Guidance for Reporting Involvement of Patients and the Public) checklist. If the public were not involved in any aspect of the research add: ``No public involvement in any aspect of this research''.}

% Only for journal Nursing Reports
%\guidelinesstandards{Please add a statement indicating which reporting guideline was used when drafting the report. For example, ``This manuscript was drafted against the XXX (the full name of reporting guidelines and citation) for XXX (type of research) research''. A complete list of reporting guidelines can be accessed via the equator network: \url{https://www.equator-network.org/}.}

% Only for journal Nursing Reports
%\guidelinesstandards{Please add a statement indicating which reporting guideline was used when drafting the report. For example, ``This manuscript was drafted against the XXX (the full name of reporting guidelines and citation) for XXX (type of research) research''. A complete list of reporting guidelines can be accessed via the equator network: \url{https://www.equator-network.org/}.}



%%%%%%%%%%%%%%%%%%%%%%%%%%%%%%%%%%%%%%%%%%
%% Optional

%% Only for journal Encyclopedia
%\entrylink{The Link to this entry published on the encyclopedia platform.}


%%%%%%%%%%%%%%%%%%%%%%%%%%%%%%%%%%%%%%%%%%
%% Optional
%%%%%%%%%%%%%%%%%%%%%%%%%%%%%%%%%%%%%%%%%%
\begin{adjustwidth}{-\extralength}{0cm}

%\printendnotes[custom] % Un-comment to print a list of endnotes


\reftitle{References}
\bibliography{mybibfile.bib}

% If authors have biography, please use the format below
%\section*{Short Biography of Authors}
%\bio
%{\raisebox{-0.35cm}{\includegraphics[width=3.5cm,height=5.3cm,clip,keepaspectratio]{Definitions/author1.pdf}}}
%{\textbf{Firstname Lastname} Biography of first author}
%
%\bio
%{\raisebox{-0.35cm}{\includegraphics[width=3.5cm,height=5.3cm,clip,keepaspectratio]{Definitions/author2.jpg}}}
%{\textbf{Firstname Lastname} Biography of second author}

%%%%%%%%%%%%%%%%%%%%%%%%%%%%%%%%%%%%%%%%%%
%% for journal Sci
%\reviewreports{\\
%Reviewer 1 comments and authors’ response\\
%Reviewer 2 comments and authors’ response\\
%Reviewer 3 comments and authors’ response
%}
%%%%%%%%%%%%%%%%%%%%%%%%%%%%%%%%%%%%%%%%%%
\PublishersNote{}
\end{adjustwidth}


\end{document}
